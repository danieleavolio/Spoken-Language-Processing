\documentclass{Interspeech2024}
\usepackage{booktabs}

\title{SLP: Native Language Identification Challenge}
% ADD authors


% the order of authors here must exactly match the order entered into the paper submission system
% note that the COMPLETE list of authors MUST be entered into the paper submission system at the outset, including when submitting your manuscript for double-blind review
\name[]{Daniele}{Avolio}
\name[]{Eduardo}{Rodrigues}

\email{danieleavolio14@gmail.com,eduardo.paz@tecnico.ulisboa.pt}

\keywords{Speech Recognition, Speech Recognition, Native Language Identification}

\begin{document}

\maketitle
  
\section{Introduction}

Speech recognition and classification have become essential technologies in
human-computer interaction, offering natural and intuitive interfaces for
various devices and applications. This study focuses on simulating a native
language identification challenge, where participants receive training,
development, and evaluation datasets, alongside baseline systems for closed-set
identification of the native language of foreign English speakers. The target
languages include Chinese, German, Hindi, and Italian.

The objective is to develop the most effective native language identification
system. During the initial phase, participants are required to understand and
enhance a baseline system utilizing MFCC features and GMM models. In the
subsequent phase, they will explore modern systems based on self-supervised
learning. The project encourages innovation through the modification and
combination of different systems to achieve optimal performance. Preliminary
results demonstrate significant improvements in accuracy, showcasing the
potential of advanced techniques in enhancing native language identification.

\section{Classical models based on conventional features}

This section describes the basic system, which consists of MFCC feature
extraction followed by GMM classification. MFCC feature extraction includes
optional components such as Delta and Double-delta calculation, Shifted Delta
Cepstrum (SDC), Voice Activity Detection (VAD) and Cepstral mean and variance
normalisation (CMVN). 

\subsection{MFCC Feature Extraction}

At the heart of our baseline system lies the Mel-Frequency Cepstral
Coefficients (MFCC) feature extraction process. The \textbf{feat\_extract}
function encapsulates this process, providing a flexible framework for
extracting MFCC features from audio files. Users can customize various
parameters such as the number of MFCC coefficients, delta computation, Shifted
Delta Cepstrum (SDC), Voice Activity Detection (VAD), and Cepstral Mean
Variance Normalization (CMVN) based on their requirements.

\subsection{Optional methods for MFCC feature extraction}

The optional methods for MFCC feature extraction include computing deltas,
Shifted Delta Cepstrum (SDC), Voice Activity Detection (VAD), and Cepstral Mean
Variance Normalization (CMVN). These methods enhance the quality of the extracted
features and improve the performance of the language identification system.
However, we didn't actually get very good results so probably there is something
wrong with the implementation, but we couldn't manage to find what could have been wrong.
Surely, we tested some combination of methods, and the most promising one was the
VAD, followed by the CMVN.


\subsection{Model Training}

We employ a supervised learning approach to train language identification
models using Gaussian Mixture Models (GMMs). For each native language in the
dataset, we train a separate GMM model using the extracted features. The models
learn the underlying distributions of features specific to each language and use
this information to classify new audio samples. 
The training process was quite fast, and we could train the models in a few
minutes. We used the \textbf{train100} datasets to train the models, and we
used the \textbf{dev} dataset, of course, to evaluate them. We even tried to use the \textbf{train}
datasets, but it was taking too long to run some tests on it. On the \textit{System Optimization} section we will talk more about the training process.


\subsection{Evaluation}
To assess the performance of the language identification system, we evaluate it
on a development set. This set contains audio samples with known ground truth
labels. We measure metrics such as accuracy, precision, recall, and F1-score to
quantify the system's performance across different languages. However, the
result were not satisfactory, achieving a score of roughly 0.6 in the Kaggle
competition. We believe that the problem was in the feature extraction, but we
couldn't manage to find what was wrong. We tried to use different combinations
of methods, but the results were always the same. We also tried to use the
\textbf{train} datasets, but even with it the results were kind of the same, so probably 
the problem was in the implementation.

\subsection{System Optimization}
Based on the evaluation results, we fine-tune the system parameters and explore
techniques for improving classification accuracy. This may involve adjusting
the number of Gaussian components in the GMMs, optimizing feature extraction
parameters, or experimenting with alternative machine learning algorithms. In
particular, we tried to estimate the best number of components for the GMMs
using the Bayesian Information Criterion (BIC) and the Akaike Information
Criterion (AIC). The process was very slow, so we did it only the first time,
and then we used the best number of components for the following experiments.

\section{Native Language Identification with Pre-trained Models}
In this section, we explore two approaches to NLI using
pre-trained models: x-vector based and self-supervised learning (SSL) based
methods. This is based on the second part of the project, where we were guided
to use the s3prl toolkit to implement the SSL approach and to use pretrained 
speaker embeddings with the x-vector based approach. We will discuss the results
we got from the x-vector based approach and the problems we had with the SSL
approach.

\subsection{X-vector Based Approach}

X-vectors, derived from deep neural networks trained for speaker
identification, have emerged as powerful embeddings for various speech
processing tasks. In this section, we delve into the process of extracting
x-vectors from audio data using pre-trained models. We demonstrate how to train
a simple linear SVM classifier on top of these embeddings and evaluate its
performance on a development set. Additionally, we experiment with alternative
models like \textbf{RandomForest, GradientBoosting, Logistic Regression, and KNN}. The
x-vector based approach yielded promising results on the development set.

\begin{table}[h!]
  \centering
  \begin{tabular}{lcccc}
    \hline
                          & \textbf{Precision} & \textbf{Recall} & \textbf{F1-Score} & \textbf{Support} \\
    \hline
    \textbf{CHI}          & 0.76               & 0.87            & 0.81              & 39               \\
    \textbf{GER}          & 0.72               & 0.64            & 0.67              & 44               \\
    \textbf{HIN}          & 0.96               & 0.94            & 0.95              & 47               \\
    \textbf{ITA}          & 0.70               & 0.70            & 0.70              & 46               \\
    \hline
    \textbf{Accuracy}     &                    &                 & 0.78              & 176              \\
    \textbf{Macro Avg}    & 0.78               & 0.78            & 0.78              & 176              \\
    \textbf{Weighted Avg} & 0.78               & 0.78            & 0.78              & 176              \\
    \hline
  \end{tabular}
  \caption{Precision, Recall, F1-Score and Support for different languages}
  \label{table:metrics}
\end{table}

The best model was the Logistic Regression, with an overall score of 0.87.
Compared to the GMM model, the x-vector based approach achieved a significant
improvement in accuracy, precision, recall, and F1-score metrics. Note that
this was the best result obtained using the transformation function
\textbf{spkrec-ecapa-voxceleb} that should perform worse than
\textbf{lang-id-voxlingua107-ecapa}. We believe that the results could be
improved by using the \textbf{train} datasets, but the transformation function 
\textbf{lang-id-voxlingua107-ecapa} was taking too long to run.

\subsection{Self-Supervised Learning (SSL) Approach}
The s3prl toolkit offers a suite of self-supervised pre-trained models for
speech processing tasks. In this section, we explore how to leverage SSL models
for NLI by fine-tuning them on our dataset. We discuss the downstream task
setup and the architecture of the simple model used for NLI. Furthermore, we
provide instructions for running scripts to utilize SSL models and save results
for evaluation. The SSL approach using the s3prl toolkit provided competitive
results:
\begin{table}[htbp]
  \centering
  \caption{Language Classification Metrics}
  \begin{tabular}{lccc}
    \toprule
    Language & Precision (\%) & Recall (\%) & F1-score (\%) \\
    \midrule
    CHI      & 76             & 87          & 81            \\
    GER      & 72             & 64          & 67            \\
    HIN      & 96             & 94          & 95            \\
    ITA      & 70             & 70          & 70            \\
    \midrule
    Overall  &                &             & 78.41         \\
    \bottomrule
  \end{tabular}%
  \label{tab:metrics}%
\end{table}%

However we had some problems to run this, because I couldn't get this to work
on Windows (I tried to run it on WSL, but it didn't work) and I don't know why.
Unfortunately, I didn't have time to try to do more tests so we just kept the
results we got from the lab when we did manage to run it throug Eduardo's
computer. Since it's a very powerful tool, we believe that the results could be
much better than the ones we got from the x-vector based approach if we had more time to 
work on it.

\section{Conclusion}
The conclusion of this study highlights the evolution and diversity of
approaches to native language identification, which are essential for a wide
range of applications in human-computer interaction, language processing, and
security.

Classical models, based on conventional features such as mel frequency cepstral
coefficients (MFCC) and Gaussian mixture models (GMM), lay a solid foundation
for native language identification. However, it's clear that new approaches
like x-vectors and self-supervised learning (SSL) models offer significantly better
performance, as demonstrated by the results obtained in this project.

Preliminary results reveal that the use of x-vectors and SSL models results in
promising precision, recall and F1-score metrics, effectively competing with
classical models in several target languages.

It's clear that the future of native language identification lies in the
exploration and integration of advanced techniques like x-vectors and SSL models,
but it's also important to consider the computational resources and expertise
required to implement these approaches effectively. Actually, the implementation 
being guided made the process easier and it would have been much harder to do it
manually and in this short amount of time.

A very important note is that we did all our tests only on the
\textbf{train100} datasets due to time constraints. We believe that the results
could be improved by using the \textbf{train} datasets, but it would have taken
too long to run the tests. Maybe finding another transformation function 
different from the \textbf{2} we used could have improved the results as well,
going more specifically to the languages we were trying to identify.
\end{document}
